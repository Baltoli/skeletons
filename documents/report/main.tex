\documentclass[journal]{IEEEtran}

\usepackage{blindtext}
\usepackage{graphicx}

\ifCLASSINFOpdf
  % \usepackage[pdftex]{graphicx}
  % declare the path(s) where your graphic files are
  % \graphicspath{{../pdf/}{../jpeg/}}
  % and their extensions so you won't have to specify these with
  % every instance of \includegraphics
  % \DeclareGraphicsExtensions{.pdf,.jpeg,.png}
\else
  % or other class option (dvipsone, dvipdf, if not using dvips). graphicx
  % will default to the driver specified in the system graphics.cfg if no
  % driver is specified.
  % \usepackage[dvips]{graphicx}
  % declare the path(s) where your graphic files are
  % \graphicspath{{../eps/}}
  % and their extensions so you won't have to specify these with
  % every instance of \includegraphics
  % \DeclareGraphicsExtensions{.eps}
\fi

\hyphenation{op-tical net-works semi-conduc-tor}

\begin{document}

\title{Detection and Optimisation of Parallel Skeletons}
\author{Bruce Collie (Trinity Hall)}

\markboth{Modern Compiler Design, Part III Computer Science 2016-17}{}

\maketitle

\begin{abstract}

We want to detect parallel skeletons in C code.

\end{abstract}

\begin{IEEEkeywords}
parallel skeletons, auto-parallelisation, optimisation
\end{IEEEkeywords}

\IEEEpeerreviewmaketitle

\section{Introduction}

\section{Conclusion}

\ifCLASSOPTIONcaptionsoff
  \newpage
\fi

\end{document}
